\pdfoutput=1

\documentclass{l4proj}
\usepackage{url}
\usepackage{breakurl}
\usepackage[breaklinks]{hyperref}
\def\UrlBreaks{\do\/\do-}
\usepackage[T1]{fontenc}

\begin{document}
\title{Automatically determining next semantic version of a project by \\
  inspecting changes to the codebase}
\author{Christian Shtarkov}
\date{\today}
\maketitle

\begin{abstract} In software development, assigning a version number
to each release is crucial. It allows customers and other developers
to identify that a software package has changed from a previous
iteration. It is especially important when considering libraries and
frameworks, where changes can greatly affect applications built on top
of them. \\ Different policies exist dictating how to assign version
numbers, one of the most popular being \textit{Semantic
Versioning}\cite{SemanticVersioning}, where the version gives
information about how exactly the software package has changed. This
allows for better estimation whether a new version of a library or a
framework would result in incompatibilities.
\\\\
Regardless of
policy, assigning version numbers is typically done manually and
leaves room for human mistakes or deviations from the established
scheme. We propose the tool \textit{autobump} that automatically
inspects changes to a codebase and proposes the next version number
according to semantic versioning. This reduces friction when releasing
packages, and encourages more frequent, even fully automated,
releases.
\end{abstract}

\educationalconsent

\tableofcontents

%% Contents
\chapter{Introduction}
\pagenumbering{arabic}

\section{Versioning policies}

There are different policies that dictate how versions of software
projects change with new releases. Popular ones include:

\begin{itemize}
\item Using the revision control number - typically only used
internally and can change multiple times per day. This can be a single
number increasing incrementally, such as the revision in Subversion,
or some hash.
\item Code names - most frequent in consumer-facing software, such as
\textit{macOS Sierra}. Code names are usually picked for marketing
purposes, rather than telling users something meaningful about the
software itself.
\item Semantic Versioning (Semver)\cite{SemanticVersioning}.
\end{itemize}

Semver reveals information about how the software package has changed
from a previous iteration by dividing the version into three
independent numbers:
\begin{center}
\textbf{MAJOR.MINOR.PATCH}
\end{center}

\begin{itemize}
\item \textbf{MAJOR} increments when incompatible changes to the API
have been made.
\item \textbf{MINOR} increments when new features has been added in a
backwards-compatible manner.
\item \textbf{PATCH} increments when bugs were fixed in a
backwards-compatible manner.
\end{itemize}

Semver is especially useful for software meant for use by other
software, such as libraries and frameworks. When a new version of a
library is released, someone developing an application based on that
library can quickly judge how the update affects their application.

If only the minor or patch numbers have changed, then the application
should work fine with the new version. If, however, the major number
has increased, then further work on part of the application developer
may be needed to make it compatible again with the library.

This scheme also helps with automatic dependency management systems,
like ones found in package managers. A developer can specify that
their application relies on version \textit{2.4} of a particular
library. Semver guarantees that versions of the form \textit{(2.x | x
 > 4)} will contain all features found in \textit{2.4} and be backwards-compatible.
If later on \textit{2.4.1} or \textit{2.5} come along, the
library can be automatically upgraded without breaking the
application, with the added benefit of fixing bugs.

Semver only provides these benefits when it is applied consistently.
If it is applied haphazardly, then two versions cannot be compared
reliably and decisions made by simply examining version numbers are
risky. A good example of that is the programming language PHP, where
the behaviour comparison operator was changed in a patch release, and
then restored in a minor release. Evaluating \texttt{'0xABCdef' == '
0xABCdef'} produces \texttt{true} in PHP 5.0.2 but \texttt{false} in
PHP 5.0.3. In PHP 5.2.0, it starts to produce \texttt{true} again.
Applications relying on the behaviour of this operator will unexpectedly
break, because there was an implicit (not reflected in the version number)
breaking change.

\section{Automatically determining next semantic version}
The goal of this project was to build a tool that can intelligently propose
the next semantic version of a software package by examining the codebase.

The tool \textit{autobump} can compare two commits in a version controlled
project and if the version number of the earlier commit is known, it can determine
what the version number of the later commit should be. It adheres to the formal
rules specified by Semver\cite{SemanticVersioning} and detects breaking changes,
added features and bugfixes.

%% TODO: A paragraph on language support.

The tool can be used standalone as a sanity check for the developer, or integrated
as part of a continuous deployment mechanism so that correct version numbers are
automatically assigned to releases. Using it instead of assigning versions manually
also leaves less room for human errors, such as presumably the one that caused PHP
to have different behaviour in two consecutive patch releases.

%% TODO: Also generate changelogs?

\section{Similar work}
Automatically assigning version numbers is not a new idea and projects
exist that already do that to some extent.

%% https://github.com/GitTools/GitVersion
%% https://github.com/semantic-release/semantic-release
%% https://github.com/endjin/Endjin.Assembly.ChangeDetection

%% Bibliography
\bibliographystyle{plain}
\bibliography{dissertation}

\end{document}