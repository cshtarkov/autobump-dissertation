\pdfoutput=1

\documentclass{l4proj}
\usepackage{url}
\usepackage{breakurl}
\usepackage[breaklinks]{hyperref}
\def\UrlBreaks{\do\/\do-}

\begin{document}
\title{Automatically determining next semantic version of a project by \\
  inspecting changes to the codebase}
\author{Christian Shtarkov}
\date{\today}
\maketitle

\begin{abstract} In software development, assigning a version number
to each release is crucial. It allows customers and other developers
to identify that a software package has changed from a previous
iteration. It is especially important when considering libraries and
frameworks, where changes can greatly affect applications built on top
of them. \\ Different policies exist dictating how to assign version
numbers, one of the most popular being \textit{Semantic
Versioning}\cite{SemanticVersioning}, where the version gives
information about how exactly the software package has changed. This
allows for better estimation whether a new version of a library or a
framework would result in incompatibilities.
\\\\
Regardless of
policy, assigning version numbers is typically done manually and
leaves room for human mistakes or deviations from the established
scheme. We propose the tool \textit{autobump} that automatically
inspects changes to a codebase and proposes the next version number
according to semantic versioning. This reduces friction when releasing
packages, and encourages more frequent, even fully automated,
releases.
\end{abstract}

\educationalconsent

\tableofcontents

%% Contents
\chapter{Introduction}
\pagenumbering{arabic}

%% Bibliography
\bibliographystyle{plain}
\bibliography{dissertation}

\end{document}